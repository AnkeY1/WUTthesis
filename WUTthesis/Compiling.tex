\section{编译方式}

\par {\LaTeX}的编译方式有多种,其中具体的区别可参考相关资料,这里推荐使用\texttt{xelatex},该编译方式可以直接处理UTF8编码的字符。因为需要基于\texttt{biblatex}生成参考文献,完整的编译分为如下四步:
\begin{itemize}
\item \texttt{xelatex Thesis.tex}
\item \texttt{biber Thesis}
\item \texttt{xelatex Thesis.tex}
\item \texttt{xelatex Thesis.tex}
\end{itemize}
不过在具体的论文撰写过程中,当暂时不关心参考文献的生成时,可以仅用\texttt{xelatex Thesis.tex}编译文档来查看排版效果。{\color{red} 有时编译后发现论文中的公式或图表的链接部分显示“??”,这时只需要再编译一次即可。编译时,可能由于字体大小替换而出现一些警告,这个时候可以忽略,不影响最终编译。}




