\section{{\LaTeX}安装}

\par 下面列出一些{\LaTeX}套装以及相应的适用操作系统和官方网站
\begin{itemize}
\item MiKTeX (Microsoft Windows, \url{https://miktex.org/});
\item MacTeX (macOS, \url{https://www.tug.org/mactex/});
\item CTeX (Microsoft Windows, Chinese TeX, \url{http://www.ctex.org/HomePage})\footnote{这里需要区分CTeX套装和{\CTeX}宏集。};
\item TeX Live (Unix-like/Microsoft Windows/macOS, \url{http://tug.org/texlive/})。
\end{itemize}
如果在Ububtu(Linux的一种发行版)操作系统下,可以直接通过命令\texttt{sudo apt-get install texlive-full}安装。因为TeX Live具有最广泛的操作系统适用性,所以也自然成为最为推荐的{\LaTeX}套装,其相应的编辑器为 \href{http://www.tug.org/texworks/}{TeXworks}(Linux系统下需要单独安装并通过配置指定已安装的Tex Live可执行文件路径。)\footnote{TeXworks是一种开源的、跨平台、多语言支持的TeX/LaTeX编辑器。其他类似的编辑器包括 \href{https://www.texstudio.org/}{TeXstudio}、\href{https://www.xm1math.net/texmaker/}{Texmaker} 等。}。另外,需要指出的是,CTeX套装虽然针对中排版的{\LaTeX}系统,由于以下两个原因:
\begin{itemize}
\item 截止到2020年2月的最新版本,CTeX套装中并没有包含\texttt{biblatex-gb7714-2015}宏包(该宏包用于设置中文参考文献样式);
\item 其配套的WinEdt\footnote{\href{http://www.winedt.com/}{WinEdt} 是闭源软件,只适用与Microsoft Windows平台,不过用户可以免费使用该软件。}编辑器中并没有发现对应于\texttt{biblatex}宏包的参考文献处理程序\texttt{biber}按钮;
\end{itemize}
不建议使用。而实际上,由于上述原因,\texttt{WUTthesis}也未在CTeX套装上测试成功。不过新版CTeX套装预计会在2020年4月之前发布,变动会很大,我们拭目以待。当然,这里也鼓励喜欢钻研{\LaTeX}技术用户尝试解决上述两个原因造成的在使用\texttt{WUTthesis}过程出现的问题。











