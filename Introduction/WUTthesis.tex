\section{\texttt{WUTthesis}}



\par 正由于{\LaTeX}在诸多方面的优势,国内越来越多的大学都开始鼓励采用其作为学位论文的排版系统,尤其是对于理工科专业。针对各大学学位论文的格式要求,一些校友们也已经制作了相应的学位论文{\LaTeX}模板。这里值得特别提出的是,在制作并推广学位论文{\LaTeX}模板方面,已经有两位武汉理工大学的校友作出过努力,他们的个人信息、作品名称,作品网址列出如下:
\begin{itemize}
\item 胡卫谊,2010年硕士研究生毕业,武汉理工大学计算机科学与技术学院,
\begin{itemize}
\item 作品:《武汉理工大学学位论文{\LaTeX}模板\texttt{WHUTthesis}》(研究生),
\item 作品网址:\url{https://code.google.com/archive/p/whutthesis/},
\item 作者邮箱:\texttt{whutthesis@gmail.com};
\end{itemize}
\item 曹宇,2015年本科毕业,武汉理工大学交通学院,
\begin{itemize}
\item 作品:《武汉理工本科论文{\LaTeX}模板》,
\item 作品网址:\url{https://github.com/tsaoyu/WHUT-LaTeX-bachelor},
\item 作者邮箱:\texttt{thesis@tsaoyu.com}。
\end{itemize}
\end{itemize}
但可惜的是,到目前位置(2020年2月),他们的工作并没有得到官方的认可,反而在提交学位论文进行反剽窃检查时,被要求提供doc格式的文档\footnote{此处省去太多吐槽。}。胡卫谊制作的模板因为长时间没有维护,再加上该模板在结构等方面存在一些需要改进和提升的地方,所以有必要重新制作新的模板,而本文就是针对这一目的介绍一种新的武汉理工大学研究生学位论文{\LaTeX}模板:\texttt{WUTthesis}。该新模板是根据《武汉理工大学博士、硕士学位论文撰写、印刷格式的统一要求(2009年7月修订)》制作的,模板在制作方面遵循以下原则:
\begin{itemize}
\item 模板要尽可能的简单,不做过多的封装,尽量将一些格式设置放在主文件的导言区。过多的封装会隐藏模板设计细节,从而影响用户对模板的理解。(这里鼓励用户在遵循学校统一格式要求和充分理解设计细节的情况下,根据自己的需求对模板进行修改。)
\item 模板的结构必须和学位论文的结构保持一致,从而可以使用户能快速且直观地理解模板的结构。
\end{itemize}


\par \texttt{WUTthesis}的改进和提升需要官方的配合,以及使用者的不断反馈。广大用户针对\texttt{WUTthesis}如果发现什么问题,或是有什么意见或建议,请联系作者本人,邮箱为\texttt{gujiayin1234@163.com}。{\color{red} 这里必须声明如下:由于\texttt{WUTthesis}格式上无法100\%地保证没有任何问题,再者,因为\texttt{WUTthesis}还未得到官方认可(到2020年2月为止),最终会要求在网上系统提交doc格式的文档,由此可能会引起的潜在种种问题,作者本人概不负责。}


